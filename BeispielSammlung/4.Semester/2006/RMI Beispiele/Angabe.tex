\documentclass{article}

\begin{document}

\title{RMI \"Ubungsbeispiele}

\author{Markus Cozowicz}

\section{Allgemein}

Der Client soll Usern die jeweilige Funktionalit\"at \"uber eine Kommando Eingabe zur Verf\"ugung stellen.\\

\noindent Einlesen von Kommandozeile:\\
\begin{verbatim}
	BufferedReader br = new BufferedReader(new InputStreamReader(System.in));
\end{verbatim}

% \section{Auslesen einer Datei}
% \section{EBay - Mitsteigern}

\section{Forum}

\subsection{Server}

Bietet Client folgende Funktionen

\begin{itemize}
	\item Auflistung von Themen
	\item Auflistung von Postings zu einem Thema
	\item Post'n eines Eintrags zu einem Thema.
	      Existiert das Thema des neuen Postings nicht, wird es zur Liste der Themen hinzugef\"ugt.
\end{itemize}

\subsection{Client}

Jeweils passende Ein-/Ausgaben zu den Server Funktionen.

%%%%%%%%%%%%%%%%%%%%%%%%%%%%%%%%%%%%%%%%%%% 

\section{Voting}

\subsection{Server}

Bietet Clients folgende Funktionen

\begin{itemize}
	\item Bereitstellung von Fragen aus einer Datei. Die Datei soll jeweils Zeilenweise Fragen beinhalten.
	\item Abstimmen f\"ur eine Frage.
	\item Bereitstellung der Fragen und den jeweils zugeh\"origen Stimmenanzahl.
\end{itemize}

\subsection{Client}

\begin{itemize}
	\item Auflistung der Fragen.
	\item Abstimmen f\"ur eine Frage.
	\item Auflistung des Wahlergebnis mit absoluter und prozentueller Stimmenanzahl.
\end{itemize}

\end{document}
